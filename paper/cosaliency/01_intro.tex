\section{Introduction}
Comparison is an indispensable task in data analysis and visualization. It often involves searching for categories (classes) with large or small changes among multiple categorical datasets.
%\footnote{The words ``labeled'' and ``categorical'' are interchangeable, and we study the quantitative data with a categorical variable.}.
Such comparison is usually achieved through juxtaposition of multiple visualizations~\cite{Gleicher18,LYi21} such as multi-class scatterplots, line and bar charts.
Regardless of the visualization type, each class is commonly encoded by a unique color. While color plays an important role in helping viewers see differences between juxtaposed views~\cite{Tominski08,Albers11,Gleicher18}, finding an appropriate color mapping scheme to ease the process for comparative visualization is a challenging and yet unexplored problem.

The most common way to colorize juxtaposed views is finding an appropriate color mapping for one  artificially selected view while judging how well it fits to the other views. Such a trial and error procedure might converge to a desirable color mapping; however, its required efforts significantly increase with the numbers of classes and views. Although existing automated color selection approaches~\cite{Chen14,Wang2018,Lu21} can alleviate the effort for single view colorization, the obtained color mapping might not be able to clearly reveal similarities or differences among multiple views. For example, the optimized assignment\cite{Wang2018} of the Tableau palette in Fig.~\ref{fig:teaser}(middle top) creates a visualization with better class separation than the one generated by random assignment in Fig.~\ref{fig:teaser}(left top), although the changed pink class is hard to be identified.
As far as we know, few existing visualization-oriented color selection tools (e.g., ColorBrewer~\cite{harrower2003colorbrewer} or Palettailor~\cite{Lu21}) allow for colorizing multi-view visualizations, let alone supporting comparisons in juxtaposed views. %by juxtaposition.
%seeing differences between juxtaposed views.

There are two simple ways to assist comparison task, one is using alpha-blending to highlight concerned classes, the other is using faceting by groups with the groups highlighted on top of all the cases~\cite{}. These two methods only cared about highlighting the concerned classes while make other classes invisible or hard to discriminate. However, user might want to explore one of the scatterplot rather than change the visualization, i.e., alpha-blending need to set all classes' opacity to 1.0 while facet need to show other classes in one visualization. As far as we know, there does not exist a method that unifies both highlighting important parts while maintaining good class separability for comparison task.

To fill this gap, we propose a comparison-driven color palette generation framework, which automatically generates appropriate color mappings for an effective side-by-side comparison of multiple categorical datasets. To achieve this goal, we propose a co-saliency model to characterize the most salient features among juxtaposed categorical visualizations that are likely to attract visual attention. We borrow the idea from the concept of image co-saliency~\cite{Jacobs10}, which was originally designed for summarizing salient differences between two similar natural images. %  by fusing image changes and single image saliency together.
In line with this, we devise our co-saliency model for easily identifying important features (e.g., changed classes) from juxtaposed categorical visualizations while maximizing the visual discrimination of classes in individual visualizations. It is achieved by fusing class importance between visualizations and class contrast within visualizations. The class contrast is based on perceptual separability with neighboring classes and with the background~\cite{Wang2018}, while
the class change is measured by combining point position change and point number change of each class, where the position change is quantified by using a perceptual distance metric,  Earth Mover's Distance (EMD)~\cite{rubner2000earth}.
That is, the classes with large importance and small class separabilities (strong overlap with another classes) are more co-salient, while the ones with small importance or large separabilities (more compact) being less co-salient.

By integrating our co-saliency model into existing data-aware color assignment and categorical data colorization tools~\cite{Wang2018, Lu21}, we can automatically select/generate color mappings that maximize co-saliency among juxtaposed visualizations. The resulted color mapping scheme makes the classes with large importance pop out from the context and attract viewers' attention,  while maximizing the perceptual separability between classes in individual visualizations. By doing so,
the major issue~\cite{Tominski12} of the juxtaposition is that humans have limited visual memory is greatly alleviated and the visual search can be done with less cognitive cost~\cite{healey1995visualizing}. Fig.~\ref{fig:teaser}(left bottom) shows the results generated by performing co-saliency based color assignment, where the changed class in red is easier to be spot than the one in Fig.~\ref{fig:teaser}(middle bottom). The pre-attentive ``pop out'' effect of teh class is further enhanced in Fig.~\ref{fig:teaser}(right bottom) by using our colorization method.

Since our method is based on previous work~\cite{Wang2018, Lu21}, we employ a carefully design for upward compatible. That is, our method can be used for both single or multiple scatterplots. Thus we provide a unified framework for scatterplot colorization, including color assignment with pre-defined palette and automatic palette generation. Our method can highlight interesting classes in single scatterplot due to the importance factor which can be manually adjusted by user while maintains the class separability.

Scatterplots are one of the most commonly used chart type for visualizing multi-class data, and it is harder to compare categorical scatterplots than line charts or bar charts, due to this, we mainly use them to evaluate our framework. For each of 36 multi-class scatterplots generated by using the method of Lu et al.~\cite{Lu21}, we produce its counterpart by changing properties (point number, point position) of several randomly selected classes. After scatterplot generation, we create the experiment data by applying different colorization method for each scatterplot pair, including two experimental methods based on our approach (\emph{C3-Palette Assignment}, \emph{C3-Palette Generation}) and four benchmark methods (\emph{Random Assignment}, \emph{Optimized Assignment}, \emph{Alpha Blending} and \emph{Palettailor}). With this dataset, we first conducted a pilot study to verify the validity of our experiment setting and then ran this user study to investigate how well our generated palettes help users to identify changed classes.
Second, we conducted another pilot study for the validation of visual discriminability task and then ran this study to explore the high efficiency of our method for class separability.
These two experiments are all executed through the Amazon Mechanical Turk (AMT) with 217 participants in total.
Last, we conducted a case study to show how our system helps for juxtaposed comparison of multiple categorical scatterplots.
The results show that our approach is able to produce color mapping optimized for supporting comparison and aligned with the state-of-the-art palettes in maximizing perceptual class separability.

We furthermore develop a web-based color design tool \footnote{\small \url{https://c3-palette.github.io/}}, using coordinated views for users to explore the relationship among multiple data with different color mapping schemes. %Inspired by volume visualization~\cite{kindlmann1998semi}, we provide a transfer function view for users to highlight different classes of interest.
%First, users can specify some classes of interest like the ones with subtle differences and our approach can generate  color mappings with the maximized co-saliency for these classes.
%Second, users might prefer specific colors for some classes based on semantics and accordingly our approach generates color mapping schemes that meet these constraints.
%Third, our approach can work in a reverse way that can generate a color mapping scheme to highlight the classes with less changes while obscuring the ones with large changes.
%Last, we extend our method to colorize small multiples like scatterplot matrices by integrating the spatial proximities between views
The main contributions of this paper are as follows:
%\vspace*{-2mm}
\begin{itemize}[noitemsep]
\setlength{\itemsep}{5pt}
  \item We propose a multi-class data visualization co-saliency model for measuring the importance of each data item shown in juxtaposed visualizations and use this metric to automatically generate color mapping schemes for effective comparisons;
  %\vspace*{-1mm}
    \item
  We provide an interactive tool that show how our approach can be used for helping visual comparison of multiple categorical scatterplots or even highlighting important classes within single scatterplot; and
  \item
   We evaluate the effectiveness of the resulting color mapping schemes in supporting both visual comparison and visual discriminability with two online user studies and a case study (Section~\ref{sec:results}).

  %\ms{I am not sure if categorical visualizations and categorical data is the right terminology to be used here. The data that we are working with is primarily quantitative with an additional variable (class label) that we use for color coding.  I personally feel like labeled data would be a better fit, or data with a categorical variable, but the latter is a bit long. Please pick one that you like and change the paper so the usage is consistent. I will not change further appearances of that, as this first needs a decision.  }
\end{itemize}
%\ms{general comment:  should \toolname~only refer to the webtool or to our technique.  I feel the later would be nicer. }

% Comparing multiple categorical datasets is a frequent task in visualization: trend tracing~\cite{Robertson08}, XX~\cite{} and XX~\cite{} are all examples.
% Different to single categorical data, however, the goal of visualizing multiple categorical data is not only to discriminate multiple categories, but also looking for changes from one set to another~\cite{Robertson08,Ondov19}.
% Both class discriminability and change perception are strongly influenced by the assigned colors~\cite{Tominski08,Lee13,Zhou16}. However, designing an appropriate a set of colors for multiple categorical datasets are tedious and time-consuming, especially when the number of categorical datasets and categories becomes higher.
