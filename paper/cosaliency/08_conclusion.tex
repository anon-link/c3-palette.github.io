\section {Conclusion}
We presented an interactive color design approach for the effective juxtaposed comparison of multiple labeled datasets.
It is built upon a novel co-saliency model, which characterizes the most co-salient features between juxtaposed labeled data visualizations while maintaining class discrimination in the individual visualizations.
We evaluated this approach in three ways: a numeric study for the class separability in each view,  an online study for its usability of detecting changes between multiple views, and a lab study with eye tracking to learn if our approach can alleviate eye movements. The results demonstrate that our produced color mapping schemes are well suited for efficient visual comparison. We further demonstrated the effectiveness of our approach for visually comparing juxtaposed line charts with a case study.


%In this study, we
Our work concentrated on juxtaposed comparisons to detect changes between multiple datasets.
Although detecting changes is a fundamental visual comparison task, its optimal color palette might not be appropriate for understanding other analytical comparison tasks (such as max delta and correlation tasks~\cite{Ondov19}. Future work needs to investigate the effectiveness and extensions of our approach for such comparison tasks. Furthermore, our approach produces colors with salient hue to highlight classes with large changes, but those colors do not visually indicate the ranking of class changes.
%\ms{not clear what you mean, do you mean the hue?}
%Thus, to explore classes with degree of changes, users have to manually tune the change threshold parameter.
It would be helpful to associate the color ordering constraint~\cite{Bujack18} with the degree of changes, so that the ranking of class changes can be shown clearly.
Last, while we only studied the interaction effect between change magnitude and different colorization methods, we plan to investigate how this effect is influenced by different types of changes, such as point number, center position and shape.
The order of rendering is critical for comparison task and we treat it simply in this paper by rendering less important classes first. But when there are multiple important large classes at same positions, the less important class might be overlapped and hard to distinct. Thus a professional render order algorithm is necessary for multi-class scatterplot rendering.