\section{Discussion \& Future Work}



\bpstart{Mark Attributes}
Our comparison-driven color palettes \textit{guide} viewers' perceptions of the classes with large differences from horizontally juxtaposed visualizations without \textit{enforcing} the difference.
Our study shows that a proper palette can improve the perceptual precision of visual comparisons, whereas mark orientation~\cite{liu2021data} and mark size~\cite{smart2019measuring} both might  have the effect. Note that the number of marks only influences the time for computing contrast measures without the quality of the generated palettes. The reason is that our contrast value is built on nearest neighbourhood graphs, which takes into account the visual density. Even the large datasets with multiple cases at the same position, our generated palettes still perform well.


   
\bpstart{Design Alternatives}
There are a few alternatives for visual comparison~\cite{Gleicher11}.  
Plotting difference between two data series is a clear way~\cite{Gleicher11} to indicate the item with the largest difference. However, it results in that other information  relevant to the data (such as the absolute values) may be absent. Moreover, it is hard to define and visualize the difference between two scatterplots in a meaningful way. 
In contrast, plotting multiple classes with facets~\cite{wickham2009elegant} can help for comparing all categorical visualizations. However, it requires large space when the number of classes is large. We will study and compare the effectiveness of different methods for comparing data with different sizes.


\section {Color Blindness}



Our work concentrated on juxtaposed comparisons to detect changes between multiple datasets, whereas its optimal color palette might not be appropriate for understanding other analytical comparison tasks (e.g., the correlation tasks~\cite{Ondov19}. Future work needs to investigate the effectiveness and extensions of our approach for such comparison tasks. Furthermore, our approach produces colors with salient hue to highlight classes with large changes, but those colors do not visually indicate the ranking of class changes.
It would be helpful to associate the color ordering constraint~\cite{Bujack18} with the degree of changes, so that the ranking of class changes can be shown clearly.


Second, while we only studied the interaction effect between change magnitude and different colorization methods, we plan to investigate how this effect is influenced by different types of changes, such as point number, center position and shape.
The order of rendering is critical for comparison task and we treat it simply in this paper by rendering less important classes first. But when there are multiple important large classes at same positions, the less important class might be overlapped and hard to distinct. Thus a professional render order algorithm is necessary for multi-class scatterplot rendering.

Last, our study only evaluated the effectiveness of our palettes with horizontal juxtaposed visualizations, while there are a few different layout methods such as vertical arrangement, mirrored arrangement, overlaid, and animation. Previous studies~\cite{Ondov19} show that animation performs well in identifying the largest difference and we will conduct studies to learn how well our palette works in this setting.




\bpstart{Study Limitation}
