\section{Introduction}
Comparison is an indispensable task in data analysis and visualization. It often involves searching for categories (classes) with large or small changes among multiple labeled datasets\footnote{The words ``labeled'' and ``categorical'' are interchangeable, and we study the quantitative data with a categorical variable.}.
Such comparison is usually achieved through juxtaposition of multiple visualizations~\cite{Gleicher18,LYi21} such as multi-class scatterplots, line and bar charts.
Regardless of the visualization type, each class is commonly encoded by a unique color. While color plays an important role in helping viewers see differences between juxtaposed views~\cite{Tominski08,Albers11,Gleicher18}, finding an appropriate color mapping scheme to ease the process for comparative visualization is a challenging and yet unexplored problem.

The most common way to colorize juxtaposed views is finding an appropriate color mapping for one  artificially selected view while judging how well it fits to the other views. Such a trial and error procedure might converge to a desirable color mapping; however, its required efforts significantly increase with the numbers of classes and views. Although existing automated color selection approaches~\cite{Chen14,Wang2018,Lu21} can alleviate the effort for single view colorization, the obtained color mapping might not be able to clearly reveal similarities or differences among multiple views. For example, the optimized assignment\cite{Wang2018} of the Tableau palette in Fig.~\ref{fig:teaser}(b) creates a visualization with better class separation than the one generated by random assignment in Fig.~\ref{fig:teaser}(a), although the changed purple class is hard to be identified.
As far as we know, few existing visualization-oriented color selection tools (e.g., ColorBrewer~\cite{harrower2003colorbrewer} or Palettailor~\cite{Lu21}) allow for colorizing multi-view visualizations, let alone supporting comparisons in juxtaposed views. %by juxtaposition.
%seeing differences between juxtaposed views.

To fill this gap, we propose a comparison-driven color palette generation framework, which automatically generates appropriate color mappings for an effective side-by-side comparison of multiple categorical datasets. To achieve this goal, we propose a co-saliency model to characterize the most salient differences among juxtaposed categorical visualizations that are  likely to attract visual attention. We borrow the idea from the concept of image co-saliency~\cite{Jacobs10}, which was originally designed for summarizing salient differences between two similar natural images. %  by fusing image changes and single image saliency together.
In line with this, we devise our co-saliency model for easily identifying changed classes from juxtaposed categorical visualizations while maximizing the visual discrimination of classes in individual visualizations. It is achieved by fusing  class changes between visualizations and class contrast
within visualizations. The class contrast is based on
perceptual class separability~\cite{Wang2018} and color contrast with the background, while
the class change is measured by using a perceptual distance metric,  Earth Mover's Distance (EMD)~\cite{rubner2000earth}.
That is, the classes with large changes and small class separabilities (strong overlap with another classes) are more co-salient, while the ones with small changes or large separabilities (more compact) being less co-salient.

By integrating our co-saliency model into existing data-aware color assignment and categorical data colorization tools~\cite{Wang2018, Lu21}, we can automatically select/generate color mappings that maximize co-saliency among juxtaposed visualizations. The resulted color mapping scheme makes the classes with large changes pop out from the context and attract viewers' attention,  while maximizing the perceptual separability between classes in individual visualizations. By doing so,
the major issue~\cite{Tominski12} of the juxtaposition is that humans have limited visual memory is greatly alleviated and the visual search can be done with less cognitive cost~\cite{healey1995visualizing}. Fig.~\ref{fig:teaser}(c) shows the results generated by performing co-saliency based color assignment, where the changed two classes in blue and red are easier to be spot than the ones in Fig.~\ref{fig:teaser}(b). The pre-attentive ``pop out'' effect of these two classes are further enhanced in Fig.~\ref{fig:teaser}(d) by using our colorization method.


Since scatterplots are the most commonly used chart type for labeled data visualization, we mainly use them to evaluate our framework. For each of 36 multi-class scatterplots generated by using the method of Lu et al.~\cite{Lu21}, we produce its counterpart by changing properties (point number, center position and shape) of several randomly selected classes. With this dataset, we first quantitatively measured the class separability  of our results using Lee et al.'s class visibility measure~\cite{lee2013perceptually}.
Next, we first conducted a pilot study to verify the validity of our experiment setting and then
ran two user studies to investigate how well our generated palettes help users to identify changed classes. One is an online study that compares our colorized results with the ones produced by the state-of-the-art palettes (e.g.,Tableau~\cite{tableau}) using optimal color assignments. The other is a lab study that shares the same setting with the online study but use eye tracking to verify if our results induce less eye movement.
Last, we conducted a case study to show how our selected palette helps for juxtaposed comparison of small multiples, respectively.
The results show that our approach is able to produce color mapping optimized for supporting comparison and aligned with the state-of-the-art palettes in maximizing perceptual class separability.

We furthermore develop a web-based color design tool \footnote{\small \url{https://c3-palette.github.io/}}, using coordinated views for users to explore the relationship among multiple data with different color mapping schemes. %Inspired by volume visualization~\cite{kindlmann1998semi}, we provide a transfer function view for users to highlight different classes of interest.
%First, users can specify some classes of interest like the ones with subtle differences and our approach can generate  color mappings with the maximized co-saliency for these classes.
%Second, users might prefer specific colors for some classes based on semantics and accordingly our approach generates color mapping schemes that meet these constraints.
%Third, our approach can work in a reverse way that can generate a color mapping scheme to highlight the classes with less changes while obscuring the ones with large changes.
%Last, we extend our method to colorize small multiples like scatterplot matrices by integrating the spatial proximities between views
The main contributions of this paper are as follows:
\vspace*{-2mm}
\begin{itemize}[noitemsep]
\setlength{\itemsep}{5pt}
  \item We propose a labeled data visualization co-saliency model for measuring the importance of each data item shown in juxtaposed visualizations and use this metric to automatically generate color mapping schemes for effective comparisons;
  %\vspace*{-1mm}
    \item
  We provide an interactive tool that show how our approach can be used for helping visual comparison of multiple labeled data; and
  \item
   We evaluate the effectiveness of the resulting color mapping schemes in supporting visual comparison with one quantitative study, two user studies and a case study (Section~\ref{sec:results}).

  %\ms{I am not sure if categorical visualizations and categorical data is the right terminology to be used here. The data that we are working with is primarily quantitative with an additional variable (class label) that we use for color coding.  I personally feel like labeled data would be a better fit, or data with a categorical variable, but the latter is a bit long. Please pick one that you like and change the paper so the usage is consistent. I will not change further appearances of that, as this first needs a decision.  }
\end{itemize}
%\ms{general comment:  should \toolname~only refer to the webtool or to our technique.  I feel the later would be nicer. }

% Comparing multiple categorical datasets is a frequent task in visualization: trend tracing~\cite{Robertson08}, XX~\cite{} and XX~\cite{} are all examples.
% Different to single categorical data, however, the goal of visualizing multiple categorical data is not only to discriminate multiple categories, but also looking for changes from one set to another~\cite{Robertson08,Ondov19}.
% Both class discriminability and change perception are strongly influenced by the assigned colors~\cite{Tominski08,Lee13,Zhou16}. However, designing an appropriate a set of colors for multiple categorical datasets are tedious and time-consuming, especially when the number of categorical datasets and categories becomes higher.
