\section {Evaluation}
\label{sec:results}

We evaluated the effectiveness of our method on supporting juxtaposed visual comparisons and the discriminability for reading scatterplots.% by applying it to scatterplots, as scatterplots are commonly used  for such comparisons.
We conducted two online controlled experiments through Amazon Mechanical Turk (AMT) with 160 participants in total, to evaluate how well our method can support people in \emph{observing changes} and \emph{visual separability} for juxtaposed categorical scatterplots:
\begin{enumerate}
%\vspace{-3mm}
\item [(i)] \emph{Spot-the-difference task}. To evaluate how well our method can support people in \emph{observing changes} for juxtaposed categorical scatterplots;
%\vspace{-3mm}
\item [(ii)] \emph{Counting class number task}. To evaluate whether our method can support the \emph{visual separability} of classes in each individual scatterplot, which is considered fundamental to juxtaposed comparison.
%%\vspace{-3mm}
%\item [(iii)] an eye tracking study to initially explore if our method helps people \emph{alleviate eye movement distance} when performing  juxtaposed comparison tasks.
%\vspace{-3mm}
%\item [(iv)] performing two case studies to show the usability of our methods for scatterplot matrix and time series data.
\end{enumerate}
%we use synthetic data for the first three studies while use real world data for the case study.

\vspace{.3em}
\noindent{\textbf{Independent Variables.}} In each of our studies, we investigated three independent variables: colorization method, change magnitude and change type.

\emph{Colorization method:} We used six different ways to colorize scatterplots: four benchmark methods (\emph{Random Assignment}, \emph{Optimized Assignment}, \emph{Alpha Blending} and \emph{Palettailor}) and two experimental methods based on our approach (\emph{C3-Palette Assignment}, \emph{C3-Palette Generation}):
\begin{itemize}

     \item C1: \emph{Random Assignment} is randomly selecting and assigning colors from Tableau-20 palette to the classes.

     \item C2: \emph{Optimized Assignment} uses the optimized assignment approach~\cite{Wang2018} for one of the two scatterplots with an input of Tableau-20 color palette.

     \item C3: \emph{Alpha Blending} is changing the alpha of each unchanged class to $0.5$ while the changed classes are stay to $1.0$ based on \emph{Optimized Assignment} result. We choose $0.5$ since we also have the discrimination task.
     \item C4: \emph{Palettailor} uses the method proposed by Lu et.al~\cite{Lu21} for single scatterplot palette generation. The palette is generated for one of the two scatterplots with the default settings.
     \item C5: \emph{C3-Palette Assignment} uses the color assignment optimization solution of Eq.~\ref{eq:cosaliency} based on Tableau-20.
     \item C6: \emph{C3-Palette Generation} uses the unified color generation and assignment optimization method, and produced the results with the parameters $\omega_0=1.0$, $\omega_1=1.0$ and $\omega_2=1.0$.
\end{itemize}
Our approach are all using the default parameters $\lambda=0.4$ and $\kappa=0$.

\emph{Change magnitude} and \emph{Change type}: While the colorization method is the primary independent variable to be investigated, we are also interested in how the effect of our methods would vary based on the level of change and the different change type between the two scatterplots. Thus we first define two types of changes that a class would have across multiple scatterplots: \emph{point number} and \emph{point position}. Then for each change type, we define three levels of change magnitude calculated using Eq.~\ref{eq:cm}: \emph{small}, \emph{medium}, and \emph{large}. (See the next paragraph for the detailed calculation.)

\vspace{.3em}
\noindent{\textbf{Scatterplot Dataset Generation.}}
The paired scatterplot datasets used in our studies were generated as follows.
First, we designed a set of multi-class scatterplots, each containing $8$ classes. Each class was generated using Gaussian random sampling and placed randomly in a $600 \times 600$ area.
Similar to~\cite{Lu21}, these classes belong to one of the four settings of varying size and density: small \& dense ($n=50, \sigma=20$), small \& sparse ($n=20, \sigma=50$),  large \& dense ($n=100, \sigma=50$), and large \& sparse (($n=50, \sigma=100$).

Then, for each scatterplot generated above, we produced its paired scatterplot by randomly choosing one or more classes and changing the positions or number of their data points. 
%We focused on two common change types: \emph{point number} and \emph{point position}.
To systematically compute the changes, we defined two variables: \emph{change ratio} and \emph{number of changed classes}.
\emph{Change ratio} defines how large the change of a type is, ranging from 0 to 1; and {number of changed classes} defines the number of classes that are changed, ranging from 1 to 3. We summarize our basic idea of data generation for each change type as below.
\begin{itemize}

     \item \emph{Point number}: For each class in the original scatterplot,  we calculated the new point number by multiplying the original number by ($1 \pm$ \emph{change ratio}). An addition means to increase the point number, which was implemented by generating the new points with the same distribution as the original class. Subtraction was achieved by randomly deleting data points from the original class.

     \item \emph{Point position}: Point position contains many types, such as class center position change and shape change. In our experiment, we use the two different position changes mentioned above. For center position change, the center of a class can be moved in a certain \emph{direction} with a specific \emph{distance}. We moved the center towards a random direction by a distance calculated by multiplying a maximal change distance ($400$ by default) by the \emph{change ratio}. For shape change, we define the shape of a class as the bounding box of its data points. We simulated a shape change of a class by modifying the density parameter of its Gaussian distribution to the opposite direction. For example, a small \& dense class ($n=50, \sigma=20$) would be changed into a small \& sparse ($n=50, \sigma=50$) class. In order to produce a new shape for a class, we first calculate the one-to-one mapping between the newly-generated class and the original class using ~\cite{kuhn1955hungarian} and then linearly interpolated the new point between each two points based on the \emph{change ratio} parameter. We randomly choose one change type when disturbing the class to be changed.
\end{itemize}
For each change type, we produced 100 candidate scatterplot pairs and then calculated the \emph{change magnitude} for each pair, and split all  pairs into three levels: \emph{small},  \emph{medium}, and \emph{large}.
Next, we randomly selected $2$ pairs from each change magnitude level for each change type and each number of changed classes. Thus in total we used $36$ paired scatterplot in each of the three studies. The detailed dataset is showed in Table.~\ref{tab:latinsquare}

\begin{table}[ht]
     \renewcommand\arraystretch{1}
     \centering
     \caption{Grouping of Datasets: $36$ datasets $\times$ $6$ conditions. C: condition; G: participant group; Position Small 1: point position change with small change magnitude for 1 changed class.}
     \label{tab:latinsquare}
     \begin{tabular}{lcccccccc}
     \hline
      & C1 & C2 & C3  & C4 & C5 & C6 \\
     
     \hline
     Dataset 1: Position Small 1 & \textbf{G1} & G2 & G3  & G4 & G5 & G6 \\
     Dataset 2: Position Small 1 & G6 & \textbf{G1} & G2 & G3  & G4 & G5 \\
     Dataset 3: Position Small 2 & G5  & G6 & \textbf{G1} & G2 & G3 & G4 \\
     Dataset 4: Position Small 2 & G4 & G5  & G6 & \textbf{G1} & G2 & G3 \\
     Dataset 5: Position Small 3 & G3 & G4 & G5  & G6 & \textbf{G1} & G2 \\
     Dataset 6: Position Small 3 & G2 & G3 & G4  & G5 & G6 & \textbf{G1} \\
     Dataset 7: Position Medium 1 & \textbf{G1} & G2 & G3  & G4 & G5 & G6 \\
     Dataset 8: Position Medium 1 & G6 & \textbf{G1} & G2 & G3  & G4 & G5 \\
     ... & & & & & & &\\
     Dataset 35: Number Large 3 & G3 & G4 & G5  & G6 & \textbf{G1} & G2 \\
     Dataset 36: Number Large 3 & G2 & G3  & G4 & G5 & G6 & \textbf{G1}  \\
     
     \hline
     \end{tabular}
     \end{table}

\subsection{Numeric Study}
\label{subsec:quantitystudy}
To evaluate whether our approach can fundamentally support the visual separability of the classes in each scatterplot, we conducted a numeric study using the \emph{class visibility metric} proposed by Kim et al.~\cite{lee2013perceptually}. We calculated each scatterplot in every pair, and used the average value as the final score of the pair. We then compared the scores across different colorization methods.

%\vspace{.3em}
%\noindent{\textbf{Results.}}

\begin{figure}[h]
\centering
\includegraphics[width=0.9\linewidth]{figures/classVisibility.pdf}
\caption{Average class visibility score of the 36 synthetic scatterplot pairs of each color mapping scheme.}
\vspace*{-3mm}
\label{fig:classVisibility}
\end{figure}

As shown in Fig.~\ref{fig:classVisibility}, \emph{Ours Generation} has the highest score on average. \emph{Ours Tableau-10} is sometimes higher than \emph{Random Tableau-10}. The two conditions based on the Tableau-20 palette have the lowest scores, and \emph{Ours Tableau-20} appears to be slightly lower than \emph{Random Tableau-20}. This might be caused by the palette used in these conditions. Since Tableau-20 consists of both saturated and de-saturated colors, \emph{Ours Tableau-20} tends to select several de-saturated colors for the classes that change less in order to make strongly changed classes more salient. Yet that might diminish the visual separability of the classes.

\begin{figure*}[t]
\centering
\includegraphics[width=1\linewidth]{figures/userResults.pdf}
\caption{Confidence interval plots and statistical tables for the online controlled experiment. Error bars represent 95\% confidence intervals. Each table shows the statistical test results of our experimental condition with every benchmark condition, including the mean with 95\% confidence interval ($\mu\sim$95\%CI), the W-value and p-value from the Mann-Whitney test, and the effect size ($d\sim$95\%CI).
}
\vspace*{-3mm}
\label{fig:userResults}
\end{figure*}

\subsection{Online Controlled Experiment}
\label{subsec:onlinestudy}
%We perform a controlled study to evaluate our approach based on the same datasets mentioned in Sec.~\ref{subsec:quantitystudy}.
%As our approach is the first attempt of color design for comparison, we analyze our color design framework by comparing visualizations generated under six different settings: \emph{Random Tableau-10, Random Tableau-20, Optimized Tableau, Ours Tableau-10, Ours Tableau-20, Ours Generation}.

To evaluate how well our approach can assist observing changes between juxtaposed categorical scatterplots, we conduct an online ``spot-the-difference'' experiment through Amazon Mechanical Turk (AMT) with 120 participants.
%We show participants a set of paired multi-class scatterplots with three different levels of change magnitude (\emph{small}, \emph{medium} and \emph{large}), and ask them to find a known number of classes that have been changed within $30$ seconds.  Error rate and time are recorded for analysis.

\vspace{.3em}
\noindent{\textbf{Hypotheses.}} We hypothesized that our approach would generally be more effective than the benchmark methods on the juxtaposed comparison tasks, and that this effect would vary based on \emph{change magnitude}.
%In this experiment, our major goal is to investigate if our co-saliency based color design formulation would affect the performance of observing changes between multiple datasets.
\begin{itemize}[noitemsep]
\setlength{\itemsep}{5pt}
    \item[\textbf{H1.}] Our color generation method (\emph{Ours Generation}) outperforms the benchmark conditions (\emph{Random Tableau-10}, \emph{Random Tableau-20} and \emph{Optimized Tableau}) on the task performance.

    \item [\textbf{H2.}] Our color assignment method using a color palette with a larger range of brightness and saturation (\emph{Tableau-20}) outperforms the benchmark conditions (\emph{Random Tableau-10},\emph{Random Tableau-20} and \emph{Optimized Tableau}) on the task performance.

    \item [\textbf{H3.}] There would be an interaction effect between colorization methods and change magnitude. Specifically, the difference between the effect of our methods (\emph{Ours Generation} and \emph{Tableau-20}) and that of the benchmark methods ((\emph{Random Tableau-10},\emph{Random Tableau-20} and \emph{Optimized Tableau})) would change based on the \emph{change magnitude} between the two scatterplots.
\end{itemize}

% For content we can refer to:
% \url{http://www.yunhaiwang.net/infoVis2020/palettailor/pdf/vis20a-sub1326-i6.pdf}
% \begin{itemize}
%     \item List the assignment methods: ours and the benchmarks.
%     \item Recap the definition of \emph{change magnitude} and \emph{change type}.
% \end{itemize}

\subsubsection{Experimental Design}
%In this experiment, each participant completed a ``spot-the-difference'' task that contains $40$ paired multi-class scatterplots.
%To colorize the paired scatterplots, we adopt the five visualization methods -- three of them are optimized or generated based on our approach (\emph{Ours Tableau-10}, \emph{Ours Tableau-20} and \emph{Ours Generation}), while the other two are random ones  (\emph{Random Tableau-10} and \emph{Random Tableau-20}).

\vspace{.3em}
\noindent{\textbf{Task \& Measures.}}
In this experiment, each participant was asked to perform a \emph{spot-the-difference} task. Inspired by the Spot the Difference game where one needs to compare a pair of similar pictures to detect their differences~\cite{Fukuba2009}, we asked participants to identify all the classes that have been changed in two scatterplots. At the beginning of each trial, the number of changed classes was provided. Each participant was asked to select all the changed classes by clicking the points belonging to these class in either of the scatterplots.

For each participant, we measured the \emph{time} taken for each trial, and counted the  errors ($0/1$) indicating whether the actual changed classes are aligned with the participant's response. Note that if any of the changed classes was mistakenly identified, the trial would be considered as ``wrong'' (1).

While the participant was instructed to do the task ``\emph{as accurately as possible}'', we set a 30-second time limit for each trial. If the participant could not find all the changed classes during the time limit, they were directed to the next trial.
This was done since we observed from the pilot study that when participants spent too much time on a single trial, they may decide to quit (which will lead to an incorrect answer) or to spend more time till they get the correct answer (which will lead to increasing time spent on the trial). This subject decision would add noise to our measurements. Thus we added a 30-second time limit, which was informed by our pilot study, where over 70\% correct trials were completed within $30$ seconds. More details can be found in the supplementary material.

\vspace{.3em}
\noindent{\textbf{Experiment Organization.}} We tested the effects of the $6$ method conditions across $36$ paired multi-class scatterplot datasets using a \emph{between-subject} experiment design. To avoid ordering effects, where the participant would get familiar with a dataset after seeing it several times, each participant was assigned to a group and saw a specific subset of datasets under different conditions. We used a Latin Square grouping (see Table.~\ref{tab:latinsquare}) to organize the trials for each participant. Thus, there were $6$ participant groups and each of them had $40$ trials in total. See the supplementary material for more details.

In addition, some participants might apply a ``shortcut'' strategy when seeing a class that is obviously more salient than the others, especially under the \emph{Ours Generation} condition. Thus, for quality control, we added $4$ sentinels which were very simple trials with only one changed class and a large change magnitude, and we assigned a de-saturated color to the changed class that made it less salient.
%especially to avoid preferences of \emph{Our Generation} method.
%Only answers with a over $50\%$ accuracy were accepted.

Finally, to further avoid learning effects between trials, we randomly shuffled the display orders of all scatterplot pairs, and randomly placed the two scatterplots in each pair on the left or right side.

\begin{table}[ht]
\renewcommand\arraystretch{1}
\centering
\caption{Grouping of Datasets: $36$ datasets $\times$ $6$ conditions. C: condition; G: participant group}
\label{tab:latinsquare}
\begin{tabular}{lcccccccc}
\hline
 & C1 & C2 & C3  & C4 & C5 & C6 \\

\hline
Dataset 1 & \textbf{G1} & G2 & G3  & G4 & G5 & G6 \\
Dataset 2 & G6 & \textbf{G1} & G2 & G3  & G4 & G5 \\
Dataset 3 & G5  & G6 & \textbf{G1} & G2 & G3 & G4 \\
Dataset 4 & G4 & G5  & G6 & \textbf{G1} & G2 & G3 \\
Dataset 5 & G3 & G4 & G5  & G6 & \textbf{G1} & G2 \\
... & & & & & & &\\
Dataset 35 & G3 & G4 & G5  & G6 & \textbf{G1} & G2 \\
Dataset 36 & G2 & G3  & G4 & G5 & G6 & \textbf{G1}  \\

\hline
\end{tabular}
\end{table}

\vspace{.3em}
\noindent{\textbf{Pilot Study \& Power Analysis.}}
We conducted a pilot study involving 20 participants to check the experimental setup and determine the parameters, such as the time limit for a trial.
Harnessing by the pilot study, we also obtained our expected effect sizes, which were in further fed into a power analysis. With an effect size Cohen's $d$ of $0.4$, alpha level of $0.05$ and beta level of $0.8$, the power analysis suggested a minimum number of $120$ participants for the spot-the-difference task. See the supplementary material for more details.

\vspace{.3em}
\noindent{\textbf{Participants.}}
We recruited $120$ participants (20 for each group) for the experiment on Amazon Mechanical Turk.
According to our maximum completion time ($30$ seconds for each trial), we paid each participant \$$2$ for the task based on the US minimum hourly wage.
No participant claimed color vision deficiency on their informed consent.

\vspace{.3em}
\noindent{\textbf{Procedure.}}
Each participant went through the following steps in our experiment: (i) viewing a user guide of the task and completing three training trials; (ii) completing each trial as accurately as possible within $30$ seconds; (iii) providing demographic information.

\subsubsection{Results}

Following previous studies, we analyzed the results using 95\% confidence intervals, and also conducted Mann-Whitney tests to compare the differences between conditions. The non-parametric test was used due to observations of non-normally distributed data from our pilot study. In addition, we computed the effect size using \emph{Cohen's d}, i.e., the difference in means of the conditions divided by the pooled standard deviation. We used ANOVA to examine the interaction effect between variables.

%\ms{it might be worth putting the p-value here again}
Results of the online experiment are shown in Figure \ref{fig:userResults} (left side).
First, we found that \emph{Ours Generation} leads to a significantly lower error rate in task performance than all benchmark conditions, including the random methods \emph{Random Tableau-10} (\emph{$p = 0.003$}) and \emph{Random Tableau-20} (\emph{$p < 0.001$}) and the optimized assignment method \emph{Optimized Tableau} (\emph{$p < 0.001$}). \emph{Ours Generation} also leads to significantly less time compared to \emph{Random Tableau-10} (\emph{$p = 0.001$}), \emph{Random Tableau-20} (\emph{$p < 0.001$}) and \emph{Optimized Tableau} (\emph{$p < 0.001$}) (\textbf{H1} confirmed).
Second, we found \emph{Tableau-20} as well leads to a significantly lower error rate and significant less time in task performance than all the benchmark conditions (\textbf{H2} confirmed). Since \emph{Tableau-20}, compared to \emph{Tableau-10}, uses a color palette with a larger range of brightness and saturation, we expect to see no significant differences between \emph{Tableau-10} and the benchmark conditions.
%\textbf{Question: We observed significant differences yet with small effect sizes. Do we need to provide some explanations here? Also, for the power analysis, wanted to double check if we used the .4 threshold?}\ms{as long as we have the figures with the means and CIs I think it is ok to leave as is. They visually show the effect size.}\ms{instead what seems to be missing is an explanaiton of how we conducted the analysis, I assume this was an ANOVA? This should be briefly clarified at the beginning of this section. Normality check is likely not needed as we have 120 participants, no need imho to report on that.}

Finally, we did not find significant interaction effect between \emph{colorization methods} and \emph{change magnitude}, meaning that the effect of our method is not necessarily influenced by the magnitude of change between the two scatterplots (\textbf{H3} not confirmed).


\subsection{Eye-tracking Experiment}
\label{subsec:labstudy}
To further explore how our method can help people alleviate eye movement distance when doing the juxtaposed comparison task, we conducted a lab study where we measured the participants' eye movement behavior.
%We further conducted a controlled lab study with an eye tracker to analyze the usability of our method in alleviating eye movement distances for juxtaposition visualization design.
The experimental design was similar to the online study, but we set up an accompanying eye tracker to record participants' eye movement during the experiment.
We expected to see different patterns of the scanpaths across different conditions. Specifically, our methods would lead to a shorter scanpath.



\subsubsection{Experimental Design}
\vspace{.3em}
\noindent{\textbf{Apparatus. }}
We conducted the experiment with a Tobii X60 eye tracker running at 60Hz. The eye tracking data is recorded with Tobii Studio and processed by Tobii I-VT fixation filter into fixations and saccades.
All stimuli were displayed on a 24-inch monitor with a resolution of $1920 \times 1080$ at a viewing distance of approximately 55 cm.
The experiment was conducted in an indoor room with daylight lighting conditions.
%We did not fix participants' heads, but instructed them to avoid unnecessary head movements during the training period.
%In order to avoid distractions, the experimenter located outside the participant's direct view.


\vspace{.3em}
\noindent{\textbf{Measures.}}
Following previous methodologies~\cite{Goldberg1996,Goldberg2010,Fua2017}, we measured the scanpath length to evaluate the efficiency of the visual comparison process: a longer scanpath distance indicates lower efficiency.
The scanpath was defined as the summation of distances between every two ordered fixations.
In our implementation, only scanpath length inside the area-of-interest (the paired scatterplots in each trial) will be measured.


\vspace{.3em}
\noindent{\textbf{Participants. }}
We recruited $24$ participants in total, including $14$ males and $10$ females. All the participants were volunteers from the local university majored in Computer Science, aging from $21$ to $27$.
No participant claimed  color vision deficiency on their informed consent.


\vspace{.3em}
\noindent{\textbf{Procedure. }}
The major procedure is similar to the online study. In addition, all  participants were received an explanation of the eye-tracking recording and were instructed to avoid unnecessary head movements in the training phase. Before the formal experiment, we asked the users to look at calibration dots on the screen to calibrate the eye tracker.

\subsubsection{Results}
%
%\begin{wrapfigure}{r}{0.45\columnwidth}
%\vspace{-8pt}
%\centering
%\includegraphics[width=0.45\columnwidth]{figures/labstudy.pdf}
%\vspace*{-6mm}
%\caption{}
%\vspace*{-6mm}
%\label{fig:lap}
%\end{wrapfigure}
Through this study we found that first the task performance results (time and error) are aligned with those in our online experiment (see supplemental materials for more details). Second, as shown in Figure \ref{fig:scanpath} (a), we found a slight trend that our methods (\emph{Ours Generation}, \emph{Ours Tableau 20} and \emph{Ours Tableau 10}) leads to a shorter scanpath on average than the benchmark conditions (\emph{Random Tableau-10} and \emph{Random Tableau-20}) and \emph{Optimized Tableau}). From Figure \ref{fig:scanpath} (b,c), where we plot the participant's scanpaths of two example trials, we can see that the participant had longer gazes and more eye movements across the two scatterplots under \emph{Random Tableau-20}, compared to \emph{Ours Tableau 20}. This indicates that our approach might reduce people's cognitive load needed when performing the task.
%\textbf{TODO: include some sample scanpath visualizations and discuss the observations.}\ms{yes, good idea}\ms{what about the other results? The above explanation sounded like we also recorded time and error? If yes -- we can put it into supplementals and just say, that they supported what we found in the only study. If not, slightly re-write above to say ``instead of times and errors, we recorded ...'' instead of ``in addition ...''}

\begin{figure}[ht]
\centering
\includegraphics[width=1\linewidth]{figures/scan-path.pdf}
\caption{(a) The average scanpath length of each color mapping scheme; (b,c) representative scanpaths in our lab study with two different color mapping schemes. }
\vspace*{-3mm}
\label{fig:scanpath}
\end{figure}

\vspace{.3em}
\subsection{Discussion}

In summary, we evaluated the effectiveness of our approach against the benchmark conditions through a series of studies. We found that first, our experimental methods (\emph{Ours Generation}, \emph{Ours Tableau-20} and \emph{Ours Tableau-10}) generally support the fundamental visual separability of the classes. Second, our methods outperform the benchmark methods on juxtaposed comparison tasks, and their effects are influenced by the color variety of the input palettes, yet not necessarily influenced by the change magnitude of the two scatterplots. Third, we observed some evidences indicating that our methods might help alleviate eye movement distance when doing the comparison tasks.


Some limitations exist in our evaluation.
First, our experiment focuses on scatterplots, a single visualization type. While scatterplots are commonly used in juxtaposed comparison (e.g., correlation matrix), the effectiveness of our approach might be different for other types of visualizations such as line charts or bar charts.
Second, our experiment focuses on identifying the differences between two scatterplots, which is a simplified situation, since in real-world cases often more than two visualizations are compared.
Third, we cannot further analyze the effect of \emph{change type}, given the current study design, though we did observe some trends that for certain types of change, our methods are more effective.
That brings us to a series of more fundamental questions: how can we properly define the types of changes? What is the just noticeable change magnitude for each change type?
Further research is needed to answer these questions so that our approach can be thoroughly evaluated.






